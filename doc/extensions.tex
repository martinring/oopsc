\documentclass{style}

\title{Extending the OOPS Compiler}

\begin{document}

\maketitle

%=======================================================================================
 
\section{\texttt{TRUE} and \texttt{FALSE}}

Including the \texttt{TRUE} and \texttt{FALSE} literals is easy. First we include it in the list of keywords in \texttt{de.martinring.oopsc.lexical.Tokens}:

\begin{lstlisting}
  ...  
  val keywords = SortedSet(
    AND,BASE,BEGIN,CLASS,DO,ELSE,ELSEIF,END,EXTENDS,%\textbf{FALSE}%,IF,IS,ISA,METHOD,MOD,
    NEW,NOT,NULL,OR,PRIVATE,PROTECTED,PUBLIC,READ,RETURN,SELF,THEN,%\textbf{TRUE}%,WHILE,
    WRITE,ACCESS,ASSIGN,CLOSING_PARENTHESES,COMMA,DIVIDE,EQUAL,GREATER,
    GREATER_OR_EQUAL,LESS,LESS_OR_EQUAL,MINUS,NOT_EQUAL,OPENING_PARENTHESES,
    PLUS,SEMICOLON,TIMES,TYPE_OF)
  ...  
  case object FALSE     extends Keyword("FALSE")
  ...
  case object TRUE      extends Keyword("TRUE")
  ...

\end{lstlisting}

And now it has to be included in the parser

\begin{lstlisting}
  def literal: Parser[Expression] = positioned (
      number                         { Literal.Int( ) }
    | FALSE                          { Literal.False }
    | TRUE                           { Literal.True }
    | NULL                           { Literal.Null }
    | SELF                           { VarOrCall(new RelativeName("SELF")) }
    | BASE                           { VarOrCall(new RelativeName("BASE")) }
    | NEW ~> name                    { x => New(x) at x }
    | "(" ~> disjunction <~ ")"
    | varorcall 
    | failure("expression expected"))

\end{lstlisting}

\section{\texttt{ELSE} and \texttt{ELSE IF}}

We extend our \texttt{If} type by a new parameter \texttt{elseBody}

\begin{lstlisting}
  case class If(condition: Expression, body: List[Statement], 
                %\textbf{elseBody}%: List[Statement]) extends Statement
\end{lstlisting}

Again, we introduce two new keywords in the \texttt{Lexical}

\begin{lstlisting}
  val reserved = Set(
    "CLASS", "IS", "END", "METHOD", "BEGIN", "READ", "NEW",
    "WRITE", "IF", "THEN", "WHILE", "DO", "MOD", "SELF",
    "TRUE", "FALSE", %\textbf{"ELSE"}%, %\textbf{"ELSEIF"})%
\end{lstlisting}

Now we need new production rules in the syntax:

\begin{lstlisting}
  def statement: Parser[Statement] = 
      ...
    | "IF" ~> relation ~
      "THEN" ~ (statement*) ~
       opt(elseIf)  <~
      "END" <~ "IF" { case cond~ ~body~elseBody => If(cond, body, elseBody getOrElse Nil) }    
      ...

  def elseIf: Parser[List[Statement]] = 
      "ELSE" ~> (statement*)
    | "ELSEIF" ~> relation ~ "THEN" ~
      (statement*) ~
      opt(elseIf)   { case cond~ ~body~elseBody => List(If(cond,body,elseBody getOrElse Nil)) }
\end{lstlisting}    

We also need to adjust the output in \texttt{Output.scala} and the code generation in \texttt{Code.scala}:

\begin{lstlisting}
    case If(condition, body, elseBody) => for {
      condition <- generate(condition)
      body <- sequence(body map generate)
      elseBody <- sequence(elseBody map generate)
      elseLabel <- nextLabel
      endLabel <- nextLabel
    } yield Instructions("IF")(
      Instructions("CONDITION")(
        condition,
        R5 << ~R2 || "Get condition from stack",
        R2 <<- R1,
        R5 << (R5 === O) || "if 0 then",
        Instruction("JPC", R5, elseLabel) || "jump to END IF"),        
      Instructions("THEN")(
        Instructions("")(body :*), 
        R0 << endLabel),
      Instructions("ELSE")(
        Label(elseLabel),
        Instructions("ELSE BODY")(elseBody :*)),
      Label(endLabel))           
\end{lstlisting}

We introduce a new label \texttt{elseLabel} to which we jump if the condition is false. After the body of \texttt{THEN} we jump to \texttt{endLabel}




\end{document}